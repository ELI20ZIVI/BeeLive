
\documentclass{article}
\usepackage{graphicx} % Required for inserting images
\usepackage{fancyhdr} % Required for header and footer configuration
\usepackage[a4paper, margin=2.5cm, left=1.5cm, right=1.5cm, bottom=4cm]{geometry} % Required for setting page margins
\usepackage[T1]{fontenc}
\usepackage[default,oldstyle,scale=1]{opensans} % Utilizzo del font Open Sans
\usepackage{lipsum}
\usepackage{makeidx}
\usepackage{booktabs}
\usepackage{tabularray}
\usepackage[colorlinks=true, linkcolor=black, urlcolor=blue, citecolor=blue]{hyperref}
\usepackage{tabularx}
\usepackage{makecell}
\usepackage{enumitem} % Pacchetto per la personalizzazione degli elenchi
\usepackage{booktabs}
\usepackage{subcaption}

% Configure header and footer for the first page
\fancypagestyle{firstpage}{
    \fancyhf{} % Clear header and footer
    \renewcommand{\headrulewidth}{0pt} % Remove header rule line
    \lhead{} % Header on the left
    \chead{} % Header in the center
    \rhead{} % Header on the right
    \lfoot{} % Footer on the left
    \cfoot{\vspace{5pt}\\\hrulefill\\\vspace{10pt}\textbf{BeeLive}\\Gruppo 21} % Footer in the center
    \rfoot{\vspace{32.5pt}\\\thepage} % Footer on the right
}

% Configure header and footer for non-plain pages (second page onwards)
\fancypagestyle{nonplain}{
    \fancyhf{} % Clear header and footer
    \lhead{} % Header on the left
    \chead{} % Header in the center
    \rhead{\includegraphics[width=2cm]{Images/BeeLive-Logo.png}\\\vspace{2pt}} % Header on the right
    \lfoot{} % Footer on the left
    \cfoot{\vspace{5pt}\\\hrulefill\\\vspace{10pt}\textbf{BeeLive}\\Gruppo 21} % Footer in the center
    \rfoot{\vspace{32.5pt}\\\thepage} % Footer on the right
}

% Adjust vertical space between header and text                                    
\setlength{\headsep}{65pt} 
% Adjust vertical space between text and footer
\setlength{\footskip}{0pt} 

\title{\includegraphics[width=0.75\textwidth]{Images/BeeLive-Logo.png}\\\vspace{100pt}
\LARGE{\textbf{BeeLive\\Deliverable 2}}}
\author{Gruppo 21:\\
Cipriani Pietro, 226959\\
Orlando Dennis, 227688\\
Ziviani Elia, 228172}
\date{22 Aprile 2024}

\makeindex % Indica che vogliamo creare un indice

\begin{document}

\maketitle
\thispagestyle{firstpage} % Apply firstpage style to the first page
\clearpage

\pagestyle{nonplain} % Apply non-plain style to subsequent pages

\renewcommand{\contentsname}{Indice}
\tableofcontents

\clearpage

\section{Introduzione}

\subsection{Componenti del gruppo}
\begin{itemize}
    \item Cipriani Pietro, matricola 226959
    \item Orlando Dennis, matricola 227688
    \item Ziviani Elia, matricola 228172
\end{itemize}

\subsection{Scopo del progetto}
A Trento è difficile trovare informazioni sulla viabilità, soprattutto per tutti quegli eventi che la influenzano. BeeLive risolve questo problema perchè prevede un sistema informativo che mostra visivamente le variazioni della viabilità in città, informando gli utenti di eventuali modifiche che potrebbero colpirli direttamente.

\subsection{Link utili}
\begin{itemize}
    \item \href{https://github.com/ELI20ZIVI/BeeLive/}{Repository GitHub}
    \item \href{}{Link Apiray}
\end{itemize}
\clearpage


\section{Sezione generale}

\subsection{Strategia di Branching}

\subsection{Product Backlog}

\subsection{Definition of "Done"}



\clearpage

\section{Sezione Sprint \#1}

\subsection{Goal dello sprint}
Breve descrizione del goal dello sprint.

\subsection{Sprint Planning}
Sprint Backlog con tabella e burndown chart da integrare come sezione nel report

\subsection{Test cases}
Tabella test cases -> Solo design è ok per Sprint \#1

\subsection{Sprint review}
Commenti su aspetti del meeting: demo e discussione successiva

\subsection{Product backlog refinement}
Commenti su aspetti del meeting e variazioni al product backlog

\subsection{Sprint retrospective}
Commenti sulle conclusioni del meeting, cosa ha funzionato, cosa no, adozione di pratiche "agile", "design thinking", dinamiche di gruppo, spunti per il prossimo sprint, etc.

\end{document}
