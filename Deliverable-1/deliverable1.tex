\documentclass{article}
\usepackage{graphicx} % Required for inserting images
\usepackage{fancyhdr} % Required for header and footer configuration
\usepackage[a4paper, margin=2.5cm, left=1.5cm, right=1.5cm, bottom=4cm]{geometry} % Required for setting page margins
\usepackage[T1]{fontenc}
\usepackage[default,oldstyle,scale=1]{opensans} % Utilizzo del font Open Sans
\usepackage{lipsum} 
\usepackage{makeidx}
\usepackage{booktabs}
\usepackage{tabularray}
\usepackage{subfig} % Importa il pacchetto subfig

% Configure header and footer for the first page
\fancypagestyle{firstpage}{
    \fancyhf{} % Clear header and footer
    \renewcommand{\headrulewidth}{0pt} % Remove header rule line
    \lhead{} % Header on the left
    \chead{} % Header in the center
    \rhead{} % Header on the right
    \lfoot{} % Footer on the left
    \cfoot{\vspace{5pt}\\\hrulefill\\\vspace{10pt}\textbf{BeeLive}\\Gruppo 21} % Footer in the center
    \rfoot{\vspace{32.5pt}\\\thepage} % Footer on the right
}

% Configure header and footer for non-plain pages (second page onwards)
\fancypagestyle{nonplain}{
    \fancyhf{} % Clear header and footer
    \lhead{} % Header on the left
    \chead{} % Header in the center
    \rhead{\includegraphics[width=2cm]{Images/BeeLive-Logo.png}\\\vspace{2pt}} % Header on the right
    \lfoot{} % Footer on the left
    \cfoot{\vspace{5pt}\\\hrulefill\\\vspace{10pt}\textbf{BeeLive}\\Gruppo 21} % Footer in the center
    \rfoot{\vspace{32.5pt}\\\thepage} % Footer on the right
}

% Adjust vertical space between header and text
\setlength{\headsep}{65pt} 
% Adjust vertical space between text and footer
\setlength{\footskip}{0pt} 

\title{\includegraphics[width=0.75\textwidth]{Images/BeeLive-Logo.png}\\\vspace{100pt}
\LARGE{\textbf{BeeLive\\Deliverable 1}}}
\author{Gruppo 21:\\
Cipriani Pietro, 226959\\
Orlando Dennis, 227688\\
Ziviani Elia, 228172}
\date{28 Marzo 2024}

\makeindex % Indica che vogliamo creare un indice

\begin{document}

\maketitle
\thispagestyle{firstpage} % Apply firstpage style to the first page
\clearpage

\pagestyle{nonplain} % Apply non-plain style to subsequent pages

\renewcommand{\contentsname}{Indice}
\tableofcontents

\clearpage
\section{Descrizione dell'applicativo}
\index{Descrizione dell'applicativo} % Aggiunge una voce all'indice
La Municipalità di Trento ha richiesto la valutazione dei problemi cittadini e la proposta di soluzioni attraverso lo sviluppo di un'applicazione web ad-hoc.

Dopo un processo di ideazione, è emerso che a Trento risulta spesso difficile reperire informazioni riguardanti gli eventi che influenzano la viabilità stradale. Le informazioni disponibili sono principalmente testuali e richiedono un impegno attivo da parte dei cittadini per essere trovate, poiché attualmente manca un sistema di notifica automatica per tali modifiche.

Di conseguenza, l'idea concepita prevede lo sviluppo di un sistema informativo che consenta ai cittadini di accedere rapidamente ed intuitivamente a tutte le variazioni della viabilità, senza la necessità di una ricerca attiva.\\

Il nucleo concettuale si basa sulla rappresentazione grafica delle zone coinvolte nelle modifiche della viabilità, per facilitare la comprensione delle informazioni.\\
È inoltre previsto un sistema di notifiche push per informare i cittadini in tempo reale sugli eventi che coinvolgono l'intera città o solo le zone da loro selezionate.\\

Saranno sviluppati due ambienti distinti: Uno riservato all'amministrazione e uno accessibile ai cittadini.\\
Il primo ambiente consentirà all'amministrazione di inserire e modificare le informazioni sulle variazioni della viabilità attraverso un'interfaccia grafica intuitiva e compatibile con gli standard per la visualizzazione dei dati su mappa.\\
Il secondo sarà un'applicazione mobile installabile sui dispositivi dei cittadini, che consentirà loro di visualizzare graficamente le zone interessate dai cambiamenti, selezionare le aree di loro interesse e ricevere notifiche push in tempo reale.

Ogni variazione inserita nel sistema dall'amministrazione sarà caratterizzata dalla zona interessata, visualizzabile sulla mappa, da una descrizione che ne spiega la natura e da una stima della durata della variazione, tutte informazioni che i cittadini potranno visualizzare tramite l'applicazione mobile dal loro dispositivo.
\clearpage

\section{Mockup}
\index{Mockup} % Aggiunge una voce all'indice
\subsection{Mockup applicazione mobile}
\index{Mockup applicazione mobile} % Aggiunge una voce all'indice

\begin{figure}[htbp]
    \centering
    \subfloat[Vista iniziale all'apertura dell'applicazione]{\includegraphics[width=0.25\textwidth]{Images/Mockup1 - Mobile.png}\label{fig:immagine1}}
    \hspace{100pt}
    \subfloat[Informazioni dell'evento selezionato]{\includegraphics[width=0.25\textwidth]{Images/Mockup2 - Mobile.png}\label{fig:immagine2}}
    \caption{Mockup dell'applicazione mobile}
\end{figure}

\clearpage

\subsection{Mockup interfaccia desktop}
\index{Mockup interfaccia desktop} % Aggiunge una voce all'indice
%Immagine singola
\begin{figure}[htbp]
    \centering
    \includegraphics[width=0.75\textwidth]{Images/Mockup1 - Desktop.png}
    \caption{Mockup dell'interfaccia amministrativa}
\end{figure}
\clearpage

\section{Obiettivi}
\index{Obiettivi} % Aggiunge una voce all'indice
\clearpage

\section{Attori di sistema}
\index{Attori di sistema} % Aggiunge una voce all'indice
\clearpage

\section{Prototipo}
\index{Prototipo} % Aggiunge una voce all'indice
\clearpage

\section{Requirements}
\subsection{Requirements funzionali}
\index{Requirements funzionali} % Aggiunge una voce all'indice
\clearpage

\subsection{Requirements non funzionali}
\index{Requirements non funzionali} % Aggiunge una voce all'indice
\clearpage

\section{Grafo BPMN}
\index{Grafo BPMN} % Aggiunge una voce all'indice
\clearpage

\section{Diagramma dei casi d'uso}
\index{Diagramma dei casi d'uso} % Aggiunge una voce all'indice

% Template tabella casi d'uso

\begin{table}[htbp]
\begin{tabular*}{\textwidth}{ @{\extracolsep{\fill}} || l | p{0.8\textwidth} || }
    \hline
    Nome caso d'uso & *Nome* \\
    \hline\hline
    ID & Testo \\
    \hline
    Description & Testo \\
    \hline
    Primary actors & Testo \\
    \hline
    Secondary actors & Testo \\
    \hline
    Preconditions & Testo \\
    \hline
    Main flow & Testo \\
    \hline
    Postconditions & Testo \\
    \hline
    Alternative flows & Testo \\
    \hline
\end{tabular*}
\end{table}

\clearpage

\section{Resoconto}
\index{Resoconto} % Aggiunge una voce all'indice

\end{document}
