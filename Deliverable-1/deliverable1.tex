\documentclass{article}
\usepackage{graphicx} % Required for inserting images
\usepackage{fancyhdr} % Required for header and footer configuration
\usepackage[a4paper, margin=2.5cm, left=1.5cm, right=1.5cm, bottom=4cm]{geometry} % Required for setting page margins
\usepackage[T1]{fontenc}
\usepackage[default,oldstyle,scale=1]{opensans} % Utilizzo del font Open Sans
\usepackage{lipsum} 
\usepackage{makeidx}
\usepackage{booktabs}
\usepackage{tabularray}
\usepackage[colorlinks=true, linkcolor=black, urlcolor=blue, citecolor=blue]{hyperref}
\usepackage{tabularx}
\usepackage{makecell}

% Configure header and footer for the first page
\fancypagestyle{firstpage}{
    \fancyhf{} % Clear header and footer
    \renewcommand{\headrulewidth}{0pt} % Remove header rule line
    \lhead{} % Header on the left
    \chead{} % Header in the center
    \rhead{} % Header on the right
    \lfoot{} % Footer on the left
    \cfoot{\vspace{5pt}\\\hrulefill\\\vspace{10pt}\textbf{BeeLive}\\Gruppo 21} % Footer in the center
    \rfoot{\vspace{32.5pt}\\\thepage} % Footer on the right
}

% Configure header and footer for non-plain pages (second page onwards)
\fancypagestyle{nonplain}{
    \fancyhf{} % Clear header and footer
    \lhead{} % Header on the left
    \chead{} % Header in the center
    \rhead{\includegraphics[width=2cm]{Images/BeeLive-Logo.png}\\\vspace{2pt}} % Header on the right
    \lfoot{} % Footer on the left
    \cfoot{\vspace{5pt}\\\hrulefill\\\vspace{10pt}\textbf{BeeLive}\\Gruppo 21} % Footer in the center
    \rfoot{\vspace{32.5pt}\\\thepage} % Footer on the right
}

% Adjust vertical space between header and text
\setlength{\headsep}{65pt} 
% Adjust vertical space between text and footer
\setlength{\footskip}{0pt} 

\title{\includegraphics[width=0.75\textwidth]{Images/BeeLive-Logo.png}\\\vspace{100pt}
\LARGE{\textbf{BeeLive\\Deliverable 1}}}
\author{Gruppo 21:\\
Cipriani Pietro, 226959\\
Orlando Dennis, 227688\\
Ziviani Elia, 228172}
\date{28 Marzo 2024}

\makeindex % Indica che vogliamo creare un indice

\begin{document}

\maketitle
\thispagestyle{firstpage} % Apply firstpage style to the first page
\clearpage

\pagestyle{nonplain} % Apply non-plain style to subsequent pages

\renewcommand{\contentsname}{Indice}
\tableofcontents

\clearpage
\section{Descrizione dell'applicativo}
\index{Descrizione dell'applicativo} % Aggiunge una voce all'indice
La Municipalità di Trento ha richiesto la valutazione dei problemi cittadini e la proposta di soluzioni attraverso lo sviluppo di un'applicazione web ad-hoc.

Dopo un processo di ideazione, è emerso che a Trento risulta spesso difficile reperire informazioni riguardanti gli eventi che influenzano la viabilità stradale. Le informazioni disponibili sono principalmente testuali e richiedono un impegno attivo da parte dei cittadini per essere trovate, poiché attualmente manca un sistema di notifica automatica per tali modifiche.

Di conseguenza, l'idea concepita prevede lo sviluppo di un sistema informativo che consenta ai cittadini di accedere rapidamente ed intuitivamente a tutte le variazioni della viabilità, senza la necessità di una ricerca attiva.\\

Il nucleo concettuale si basa sulla rappresentazione grafica delle zone coinvolte nelle modifiche della viabilità, per facilitare la comprensione delle informazioni.\\
È inoltre previsto un sistema di notifiche push per informare i cittadini in tempo reale sugli eventi che coinvolgono l'intera città o solo le zone da loro selezionate.\\

Saranno sviluppati due ambienti distinti: Uno riservato all'amministrazione e uno accessibile ai cittadini.\\
Il primo ambiente consentirà all'amministrazione di inserire e modificare le informazioni sulle variazioni della viabilità attraverso un'interfaccia grafica intuitiva e compatibile con gli standard per la visualizzazione dei dati su mappa.\\
Il secondo sarà un'applicazione mobile installabile sui dispositivi dei cittadini, che consentirà loro di visualizzare graficamente le zone interessate dai cambiamenti, selezionare le aree di loro interesse e ricevere notifiche push in tempo reale.

Ogni variazione inserita nel sistema dall'amministrazione sarà caratterizzata dalla zona interessata, visualizzabile sulla mappa, da una descrizione che ne spiega la natura e da una stima della durata della variazione, tutte informazioni che i cittadini potranno visualizzare tramite l'applicazione mobile dal loro dispositivo.
\clearpage

\clearpage

\section{Obiettivi}
\index{Obiettivi} % Aggiunge una voce all'indice

Archiviazione eventi\\


\clearpage

\section{Attori di sistema e Mind map}
\index{Attori di sistema e Mind map} % Aggiunge una voce all'indice

\subsection{Attori di sistema Interni}
\index{Attori di sistema Interni}

\subsubsection{Utente anonimo}
\index{Utente anonimo} % Aggiunge una voce all'indice

\subsubsection{Utente autenticato}
\index{Utente autenticato} % Aggiunge una voce all'indice

\subsubsection{Dipendente autorizzato del Comune di trento}
\index{Dipendente del Comune di trento} % Aggiunge una voce all'indice

\subsubsection{Aziende delegate}
\index{Aziende delegate} % Aggiunge una voce all'indice
I dati che arrivano da queste aziende vengono ritenuti affidabili.

\subsubsection{Amministratore di sistema}
\index{Amministratore di sistema} % Aggiunge una voce all'indice

\subsection{Attori di sistema Esterni}
\index{Attori di sistema Esterni}

\subsubsection{Aziende delegate}
\index{Aziende delegate} % Aggiunge una voce all'indice

\subsection{Componenti di sistema Interni}
\index{Componenti di sistema Interni}
\subsubsection{Applicativo mobile}
\index{Applicativo mobile}

\subsubsection{Applicativo desktop}
\index{Applicativo desktop}

\subsubsection{Web server}
\index{Web server}

Separato in due parti\\
Ogniuna con utenti autenticati

\subsubsection{Database}
\index{Database}

\subsection{Componenti di sistema Esterni}
\index{Componenti di sistema Esterni}

\subsubsection{Autenticazione}
\index{Autenticazione}

\subsubsection{Server provider}
\index{Server provider}

\clearpage

\subsection{Mind map}
\index{Mind map}

%\begin{figure}[htbp]
%    \label{fig:Mind_map}
%    \centering
%    \includegraphics[width=0.5\textwidth]{Images/MindMap.png}
%    \caption{Mind map degli attori che interagiscono con il sistema}
%\end{figure}

Tramite la mind map è possibile avere una rappresentazione chiara degli attori che interagistocno con il sistema.\\
\clearpage

\section{Prototipo del sistema}
\index{Prototipo del sistema} % Aggiunge una voce all'indice

\subsection{Mockup applicazione mobile}
\index{Mockup applicazione mobile} % Aggiunge una voce all'indice\usepackage{makecell}

\subsubsection{Schermata principale}
\index{Schermata principale} % Aggiunge una voce all'indice
\begin{figure}[htbp]
    \label{fig:Schermata_principale_mobile}
    \centering
    \includegraphics[width=0.25\textwidth]{Images/Mockup1 - Mobile.png}
    \caption{Mockup schermata principale dell'applicazione mobile}
\end{figure}

E' rappresentata la schermata principale dell'applicazione mobile che il cittadino installerà sul proprio smartphone.\\
La schermata principale mostra una mappa della città di Trento con le zone interessate dalle variazioni della viabilità.\\
La lista sottostante la mappa riporta tutte le variazioni presenti nel sistema che si trovano nel periodo di visualizzazione, con la possibilità di selezionare una variazione per visualizzarne i dettagli.\\
\clearpage

\subsubsection{Dettagli variazione selezionata}
\index{Dettagli variazione selezionata} % Aggiunge una voce all'indice
\begin{figure}[htbp]
    \label{fig:Dettaglio_evento}
    \centering
    \includegraphics[width=0.25\textwidth]{Images/Mockup2 - Mobile.png}
    \caption{Mockup schermata detagli di un evento dell'applicazione mobile}
\end{figure}

In questo caso invece è riportata la schermata di visualizzazione dei dettagli di una variazione della viabilità.\\
E' stato selezionato il precedente evento "Incontro G7", quindi sono visualizzati tutti i dettagli relativi a tale evento.\\
Nella fattispecie sono presenti la descrizione dell'evento, la data di inizio e fine, la zona interessata e la mappa con la visualizzazione della zona interessata.\\
Inoltre è presente un pulsante per tornare alla schermata principale dell'applicazione.\\
\clearpage

\subsubsection{Mockup notifiche degli eventi}
\index{Mockup notifiche degli eventi} % Aggiunge una voce all'indice

\begin{figure}[htbp]
    \label{fig:Notifica_evento}
    \centering
    \includegraphics[width=0.25\textwidth]{Images/Mockup3 - Mobile.png}
    \caption{Mockup notifiche generate da un'evento dall'applicazione mobile}
\end{figure}

L'evento preso nell'esempio precedente era impostato per generare delle notifiche push.\\
Quella riportata nell'esempio soprastante rappresenta la notifica di inizio dell'evento "Incontro G7".\\
Interagendo con essa sarà quindi aperta l'applicazione mobile sulla schermata riportata in \hyperref[fig:Dettaglio_evento]{figura 2}, permettendo quindi la consultazione delle modifiche per intero.\\

\clearpage

\subsection{Mockup interfaccia desktop}
\index{Mockup interfaccia desktop} % Aggiunge una voce all'indice
%Immagine singola
\begin{figure}[htbp]
    \label{fig:Schermata_principale_desktop}
    \centering
    \includegraphics[width=0.75\textwidth]{Images/Mockup1 - Desktop.png}
    \caption{Mockup dell'interfaccia amministrativa}
\end{figure}

L'immagine rappresenta come dovrebbe essere strutturata l'interfaccia dell'applicativo desktop, utilizzato dagli enti pubblici con potere di amministrazione per la pobblicazione degli eventi in città che ne influenzano la viabilità.\\
L'interfaccia comprende un menu laterale per la navigazione tra le varie sezioni dell'applicativo e la schermata dove il sottomen selezionato presenta tutte le sue specifiche funzionalità disponibili.

Il sottomenù selezionato in questo caso è "Gestione Eventi", la sezione in cui è possibile inserire un nuovo evento che modifica la viabilità cittadina. Infatti sono riportati tutti i campi necessari per la creazione di un evento, come il titolo, il riassunto, la data di inizio e fine di validità dell'evento e di visibilità sull'applicativo mobile, e le categorie che caratterizzano l'evento.

L'interfaccia è strutturata in modo che ogni evento abbia la possibilità di essere composto da più sottoeventi. Nella sezione di destra è possibile visualizzare l'inderimento del sottoevento "Arrivo delle delegazioni", dove è possibile inserire il titolo, la descrizione, la data di validità e la zona interessata direttamente su mappa.\\

\clearpage

\section{Requirements Funzionali}
\index{Requirements funzionali}

\subsection{Utente anonimo tramite applicativo mobile}
\index{Utente anonimo tramite applicativo mobile} % Aggiunge una voce all'indice

\subsubsection{Deve poter visualizzare la lista degli eventi}
\index{Deve poter visualizzare la lista degli eventi} % Aggiunge una voce all'indice

\subsubsection{Deve poter filtrare gli eventi per categoria}
\index{Deve poter filtrare gli eventi per categoria} % Aggiunge una voce all'indice

\subsubsection{Deve poter nascondere eventi che non gli interessano}
\index{Deve poter nascondere eventi che non gli interessano} % Aggiunge una voce all'indice

\subsubsection{Deve poter visualizzare gli eventi su mappa}
\index{Deve poter visualizzare gli eventi su mappa} % Aggiunge una voce all'indice

\subsubsection{Deve potersi registrare}
\index{Deve potersi registrare} % Aggiunge una voce all'indice

\subsubsection{Deve potersi autenticare}
\index{Deve potersi autenticare} % Aggiunge una voce all'indice

\subsubsection{Deve poter indicare aree di interesse}
\index{Deve poter indicare aree di interesse} % Aggiunge una voce all'indice

\subsubsection{Deve poter indicare percorsi di interesse}
\index{Deve poter indicare percorsi di interesse} % Aggiunge una voce all'indice

\subsubsection{Deve poter indicare linee di bus di interesse}
\index{Deve poter indicare linee di bus di interesse} % Aggiunge una voce all'indice

\subsubsection{Deve poter indicare categorie di interesse}
\index{Deve poter indicare categorie di interesse} % Aggiunge una voce all'indice

\subsubsection{Deve poter ricevere notifiche riguardo agli eventi}
\index{Deve poter ricevere notifiche riguardo agli eventi} % Aggiunge una voce all'indice

\clearpage

\section{Requirements non funzionali}
\index{Requirements non funzionali} % Aggiunge una voce all'indice

\subsubsection{Usabilità}
\index{Usabilità} % Aggiunge una voce all'indice
Per quanto riguarda l'applicazione mobile, i cittadini devono essere in grado di utilizzarla dal primo momento in cui è intallata sul dispositivo.\\
Per invece l'applicativo desktop, precedentemente al primo utilizzo è necessaria al più una specificazione delle funzionalità per il loro completo utilizzo.\\

\subsubsection{Scalabilità}
\index{Scalabilità} % Aggiunge una voce all'indice
Il requirement di scalabilità riguarda solamente l'applicazione mobile, perchè deve essere in grado di gestire un elevato numero di utenti, che rispecchi il numero dgli abitanti di Trento.\\
L'applicativo desktop invece non ha bisogno di scalabilità, in quanto è utilizzato solo da un numero limitato di utenti.

\subsubsection{Robustness}
\index{Robustness} % Aggiunge una voce all'indice
Un guasto al webserver di visualizzazione non deve compromettere il funzionamento del webserver di gestione, e viceversa.
In caso di guasto fisico, il sistema deve essere in grado di ripristinare correttamente tutti i dati senza alcuna perdita. 

\subsubsection{Prestazioni}
\index{Prestazioni} % Aggiunge una voce all'indice
La notifica di un nuovo evento deve raggiungere tutti i dispositivi mobili, connessi ad internet e con l'applicazione installata, entro un tempo di 10 minuti

\subsubsection{Reliability}
\index{Reliability} % Aggiunge una voce all'indice
Il sistema non deve essere soggetto a errori imprevisti di alcun tipo, ed eventuali bug non devono compromettere eventuali dati sensibili presenti sul server.
Il sistema gestionale usato dagli utenti autorizzati non deve essere soggetto a vulnerabilità che ne compromettano la sicurezza. 

\clearpage

\section{Grafo BPMN}
\index{Grafo BPMN} % Aggiunge una voce all'indice
\clearpage

\section{Diagramma dei casi d'uso}
\index{Diagramma dei casi d'uso} % Aggiunge una voce all'indice

\begin{itemize}
    \item It represents a functional requirement
    \item It makes clear what you expect from a system (“what?”)
    \item It hides the behaviour of the system (“how?”)
    \item It is a sequence of actions (with variants) that produce a result observable by an actor
\end{itemize}

% Template tabella casi d'uso

\begin{table}[htbp]
    \centering
    \begin{tabularx}{\textwidth}{| l | p{0.795\textwidth} |}
        \Xhline{2pt} % Linea spessa
        Nome caso d'uso & *Nome* \\
        \Xhline{2pt} % Linea spessa
        ID & Testo \\
        \hline
        Descrizione & Testo \\
        \hline
        Attori primari & Testo \\
        \hline
        Attori secondari & Testo \\
        \hline
        Precondizioni & Testo \\
        \hline
        Flusso principale & Testo \\
        \hline
        Post-condizioni & Testo \\
        \hline
        Flussi alternativi & Testo \\
        \hline
    \end{tabularx}
    \caption{Tabella riassuntiva}
    \label{tab:tabella_use_case}
\end{table}

\clearpage

\section{Resoconto}
\index{Resoconto} % Aggiunge una voce all'indice

Segue la tabella dei riferimenti, utile alla comprensione generale delle diverse sezioni riguardanti lo specifico obiettivo.\\

\begin{table}[htbp]
    \centering
    \begin{tabularx}{\textwidth}{|X|X|X|X|X|X|}
        \Xhline{2pt} % Linea spessa
        Obiettivo & Attore & User Story & Use Case\newline diagram & Requisiti\newline funzionali & Mockup \\
        \Xhline{2pt} % Linea spessa
        1 & 2 & 3 & 4 & 5 & 6 \\
        \hline
    \end{tabularx}
    \caption{Tabella riassuntiva}
    \label{tab:tabella_tutta_pagina}
\end{table}

\end{document}
