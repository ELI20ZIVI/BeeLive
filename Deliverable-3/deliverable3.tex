
\documentclass{article}
\usepackage{graphicx} % Required for inserting images
\usepackage{fancyhdr} % Required for header and footer configuration
\usepackage[a4paper, margin=2.5cm, left=1.5cm, right=1.5cm, bottom=4cm]{geometry} % Required for setting page margins
\usepackage[T1]{fontenc}
\usepackage[default,oldstyle,scale=1]{opensans} % Utilizzo del font Open Sans
\usepackage{lipsum}
\usepackage{makeidx}
\usepackage{booktabs}
\usepackage{tabularray}
\usepackage[colorlinks=true, linkcolor=black, urlcolor=blue, citecolor=blue]{hyperref}
\usepackage{tabularx}
\usepackage{makecell}
\usepackage{enumitem} % Pacchetto per la personalizzazione degli elenchi
\usepackage{booktabs}
\usepackage{subcaption}

% Configure header and footer for the first page
\fancypagestyle{firstpage}{
    \fancyhf{} % Clear header and footer
    \renewcommand{\headrulewidth}{0pt} % Remove header rule line
    \lhead{} % Header on the left
    \chead{} % Header in the center
    \rhead{} % Header on the right
    \lfoot{} % Footer on the left
    \cfoot{\vspace{5pt}\\\hrulefill\\\vspace{10pt}\textbf{BeeLive}\\Gruppo 21} % Footer in the center
    \rfoot{\vspace{32.5pt}\\\thepage} % Footer on the right
}

% Configure header and footer for non-plain pages (second page onwards)
\fancypagestyle{nonplain}{
    \fancyhf{} % Clear header and footer
    \lhead{} % Header on the left
    \chead{} % Header in the center
    \rhead{\includegraphics[width=2cm]{Images/BeeLive-Logo.png}\\\vspace{2pt}} % Header on the right
    \lfoot{} % Footer on the left
    \cfoot{\vspace{5pt}\\\hrulefill\\\vspace{10pt}\textbf{BeeLive}\\Gruppo 21} % Footer in the center
    \rfoot{\vspace{32.5pt}\\\thepage} % Footer on the right
}

% Adjust vertical space between header and text                                    
\setlength{\headsep}{65pt} 
% Adjust vertical space between text and footer
\setlength{\footskip}{0pt} 

\title{\includegraphics[width=0.75\textwidth]{Images/BeeLive-Logo.png}\\\vspace{100pt}
\LARGE{\textbf{BeeLive\\Deliverable 2}}}
\author{Gruppo 21:\\
Cipriani Pietro, 226959\\
Orlando Dennis, 227688\\
Ziviani Elia, 228172}
\date{22 Aprile 2024}

\makeindex % Indica che vogliamo creare un indice

\begin{document}

\maketitle
\thispagestyle{firstpage} % Apply firstpage style to the first page
\clearpage

\pagestyle{nonplain} % Apply non-plain style to subsequent pages

\renewcommand{\contentsname}{Indice}
\tableofcontents

\clearpage

\section{Introduzione}

\subsection{Componenti del gruppo}
\begin{itemize}
    \item Cipriani Pietro, matricola 226959
    \item Orlando Dennis, matricola 227688
    \item Ziviani Elia, matricola 228172
\end{itemize}

\subsection{Scopo del progetto}
A Trento è difficile trovare informazioni sulla viabilità, soprattutto per tutti quegli eventi che la influenzano. BeeLive risolve questo problema perchè prevede un sistema informativo che mostra visivamente le variazioni della viabilità in città, informando gli utenti di eventuali modifiche che potrebbero colpirli direttamente.

\subsection{Link utili}
Il primo link fornito è quello della repository github. La repository è stata creata appena il corso è cominciato, infatti inizialmente è risultata utile al fine di scrivere in modo collaborativo i primi deliverable.\\
E' stato deciso di integrare in questa repository anche le fasi di sprint in quanto come team riteniamo utile la possibilità di accedere alla documentazione redatta nei primi due deliverable.\\
La repository è accessibile al seguente link: \href{https://github.com/ELI20ZIVI/BeeLive/}{Repository GitHub}\\ \\
Il secondo link fornito è quello di Apiary. Apiary è uno strumento che permette di creare e documentare in modo molto accurato e approfontido le API utilizzate nel progetto.\\
Il link di Apiary è il seguente: \href{https://beelive.docs.apiary.io/#}{Link Apiray}\\

\clearpage


\section{Sezione generale}

\subsection{Strategia di Branching}

La gestione della repository del progetto è stata gestita tramite 'branches', adeguamente create e gestite in modo da isolare le diverse funzionalità su cui abbiamo lavorato. 
Segue la lista di branch create durante questo primo sprint, con tanto di descrizione:

\subsubsection{\texttt{main}}
Il branch principale del progetto, in cui vengono integrate tutte le funzionalità completate e testate.\\
Tutte le funzionalita' all'interno di questo branch sono considerate terminate e funzionanti, pronte per essere rilasciate.\\

\subsubsection{\texttt{develop}}
Il branch di sviluppo del progetto, in cui vengono integrate tutte le funzionalità completate e testate.\\
Non e' pensato propriamente come il branch in cui vengono rilasciate le funzionalita' ma piuttosto come un branch di supporto per il branch principale, infatti in questo branch vengono prima caricate le funzionalita' dagli altri branch specifici e successivamente testate. Una volta che sono considerate funzionanti vengono integrate nel branch principale.\\

\subsubsection{\texttt{ma\_screen}}
Branch dedicato alla creazione delle schermate dell'applicativo mobile, i.e. tutte le schermate che l'utente visualizzera' durante l'utilizzo dell'applicativo.\\
Questo branch è stato integrato in \texttt{ma\_client} in quanto sua dipendenza.

\subsubsection{\texttt{ma\_client}}
Branch dedicato allo sviluppo del client mobile, i.e. l'applicazione mobile utilizzata dagli utenti comuni (autenticati e non) che intendono visualizzazione le criticita' presenti in citta'.\\
Il branch, una volta concluso lo sviluppo dell'applicativo, e' stato integrato nel branch \texttt{develop} per i test in associata con tutti gli altri moduli.

\subsubsection{\texttt{ws-fetch-eventi}}
Branch dedicato allo sviluppo del webserver pubblico, i.e. il webserver in cui sono stati implementati gli endpoint API dedicati al fetching di eventi ad opera dell'applicativo mobile.\\
\\
Una volta che il modulo e' stato completato, e' stato integrato nel branch \texttt{develop} per i test con tutti gli altri moduli.


\subsubsection{\texttt{ws-insert-event}}
Branch dedicato allo sviluppo del webserver gestionale, i.e. il webserver in cui sono stati implementati gli endpoint API dedicati all'inserimento degli eventi all'interno del sistema ad opera dell'applicativo desktop. \\
Utile allo sviluppo del modulo utilizzato dall'applicativo desktop per l'inserimento delle criticita' in citta'.
Una volta che il modulo e' stato completato, e' stato integrato nel branch \texttt{develop} per i test con tutti gli altri moduli.\\
Delle versioni intermedie di questa branch sono state usate da altre branches, in particolare da \texttt{da\_desktop}, in quanto necessaria per effettuare il testing di quest'ultima.


\subsubsection{\texttt{da\_frontend}}
Branch dedicato allo sviluppo dell'applicativo desktop, quindi l'applicativo utilizzato dagli utenti autorizzati (del comune e non) per inserire a sistema le criticità (eventi) presenti in città.\\
Una volta concluso lo sviluppo del modulo frontend, il branch è stato integrato nel branch \texttt{develop} e integrato con gli altri moduli.

\subsubsection{\texttt{preview}}
Branch dedicato alla preview dell'interfaccia grafica dei due applicativi.\\
E' stato utilizzato al momento della prima presentazione del progetto per mostrare il lavoro svolto ai committenti e per ricevere feedback su eventuali modifiche da apportare.\\
Integra dei mockup delle varie interfacce, diaa desktop che mobile, che andranno sviluppate.\\
Parte della preview è stata poi utilizzata in \texttt{da\_frontend} come punto di partenza per l'interfaccia,

\subsection{Product Backlog}


\subsection{Definition of "Done"}
La Definition of Done è un insieme di criteri utilizzati per determinare quando un'attività o una user story può essere considerata completata. Questi criteri servono a garantire che ogni elemento del lavoro raggiunga un livello di qualità accettabile per considerarlo terminato.\\
I criteri inclusi nella Definition of Done redatta per il progetto sono i seguenti:
\begin{itemize}
    \item \textbf{Codice}: Il codice deve essere rivisto e approvato attraverso un processo di code review, e deve aderire agli standard di codifica stabiliti.
    \item \textbf{Test}: Tutti i test unitari, di integrazione e funzionali devono essere scritti ed eseguiti con successo.
    \item \textbf{Documentazione}: La documentazione necessaria, come i deliverable, deve essere completa e aggiornata.
    \item \textbf{Build e Deployment}: Il codice deve essere integrato con successo nella build principale e deve essere distribuito in un ambiente di test per ulteriori verifiche.
    \item \textbf{Performance}: Il prodotto deve soddisfare i criteri di performance definiti, garantendo che le funzionalità siano efficienti e scalabili.
    \item \textbf{Conformità}: Il prodotto deve essere conforme alle normative legali e agli standard di sicurezza e qualità specifici del settore. 
\end{itemize}

\clearpage

\section{Sezione Sprint \#1}

\subsection{Goal dello sprint}
Lo scopo di questo primo sprint è stato quello di creare un prototipo funzionante dell'applicazione, che presenti le funzionalità core del nostro progetto e che quindi includa queste funzionalità:
\begin{enumerate}
\item possibilità da parte di un utente autorizzato di aggiungere eventi al sistema
\item possibilità da parte degli utenti di visualizzare gli eventi aggiunti
\end{enumerate}
È stato dunque necessario lavorare, oltre al 'backend', anche anche sull'applicativo mobile dedicato ad utenti comuni e sull'applicativo desktop, dedicato agli utenti autorizzati.

Scendendo più in dettaglio, le funzionalità che abbiamo deciso di includere nello sprint sono, per quanto riguarda la parte mobile, la visualizzazione degli eventi disponibili e dei loro relativi dettagli, anche includendo la visualizzazione su mappa.\\
Abbiamo anche inserito, anche se con bassa priorità, la funzionalità riguardante la possibilità di effettuare l'autenticazione (da parte degli utenti comuni) per poter salvare le proprie impostazioni e preferenze sul database.\\
Per quanto riguarda la parte desktop, ci siamo limitati a rendere possibile l'aggiunta degli eventi al sistema, rimandando l'autenticazione degli utenti autorizzati a sprint successivi.

\subsection{Sprint Planning}
E' riportato il product backlog per ogni user story trattata in questo primo sprint, con tutte le informazioni relative ai task da completare, a chi sono assegnati e la loro priorità.\\
\subsubsection{Aggiunta eventi da utente autorizzato}
\textbf{User story}: Da utente autorizzato, devo essere in grado di aggiungere degli eventi in modo da comunicare la loro presenza agli utenti.\\
\begin{table}[htbp]
    \centering
    \renewcommand{\arraystretch}{1.3} % Imposta lo spazio verticale delle righe
    \begin{tabularx}{\textwidth}{| X | r | r | r | r |}
        \Xhline{2pt}
        \makecell{\textbf{Nome}} & \makecell{\textbf{User story}} & \makecell{\textbf{Cosa fare}} & \makecell{\textbf{Assegnazione}} & \makecell{\textbf{Stima}} \\
        \Xhline{2pt}
        \makecell{Aggiunta\\eventi\\da utente\\autorizzato} & \makecell{Da utente autorizzato,\\devo essere in grado di\\aggiungere degli eventi in\\modo da comunicare la\\loro presenza agli utenti} & \makecell{Creazione screen (DA)\\Creazione metodo del client (DA)\\Creazione API endpoint (WS)\\Creazione metodo del DAO (WS)\\Creazione modulo gestione mappa\\Creazione event manager (WS)\\Creazione test (WS)\\Creazione test (DA)} & \makecell{Dennis Orlando\\Dennis Orlando\\Dennis Orlando\\Pietro Cipriani\\Pietro Cipriani\\Pietro Cipriani\\Elia Ziviani\\-} & \makecell{5\\3\\4\\4\\3\\2\\4\\4} \\
        \hline
    \end{tabularx}
    \caption{Product backlog user story 1}
\end{table}

\clearpage

\subsubsection{Visualizzazione eventi e dettagli da utente mobile}
\textbf{User story}: Da utente, voglio avere la possibilità di accedere ad una visualizzazione e descrizione più dettagliata di un evento in modo da conoscere meglio le specifiche di tale evento.\\
\begin{table}[htbp]
    \centering
    \renewcommand{\arraystretch}{1.3} % Imposta lo spazio verticale delle righe
    \begin{tabularx}{\textwidth}{| X | r | r | r | r |}
        \Xhline{2pt}
        \makecell{\textbf{Nome}} & \makecell{\textbf{User story}} & \makecell{\textbf{Cosa fare}} & \makecell{\textbf{Assegnazione}} & \makecell{\textbf{Stima}} \\
        \Xhline{2pt}
        \makecell{Visualizzazione\\eventi\\e dettagli\\da utente\\mobile} & \makecell{Da utente, voglio avere\\la possibilità di accedere\\ad una visualizzazione e\\descrizione più dettagliata\\di un evento in modo da\\conoscere meglio le\\specifiche di tale evento} & \makecell{Creazione screen (DA)\\Creazione API endpoint (WS)\\Creazione metodo del client (DA)\\Creazione metodo del DAO (WS)\\Estensione Event Processor\\Creazione test (WS)\\Creazione test (DA)} & \makecell{-\\Dennis Orlando\\-\\Dennis Orlando\\Dennis Orlando\\Elia Ziviani\\-} & \makecell{4\\2\\2\\2\\2\\4\\4} \\
        \hline
    \end{tabularx}
    \caption{Product backlog user story 2}
\end{table}

\subsubsection{Registrazione account utente mobile}
\textbf{User story}: Da utente voglio essere in grado di poter registrare un account in modo da poter salvare le mie impostazioni e le aree di interesse che io ho impostato.\\
\begin{table}[htbp]
    \centering
    \renewcommand{\arraystretch}{1.3} % Imposta lo spazio verticale delle righe
    \begin{tabularx}{\textwidth}{| X | r | r | r | r |}
        \Xhline{2pt}
        \makecell{\textbf{Nome}} & \makecell{\textbf{User story}} & \makecell{\textbf{Cosa fare}} & \makecell{\textbf{Assegnazione}} & \makecell{\textbf{Stima}} \\
        \Xhline{2pt}
        \makecell{Registrazione\\account\\utente\\mobile} & \makecell{Da utente voglio essere\\in grado di poter registrare\\un account in modo da\\poter salvare le mie\\impostazioni e le aree di\\interesse che io ho impostato} & \makecell{Creazione screen login (MA)\\Istanziazione casdoor\\Sincro preferenze locali (MA)\\Client: Gestione token (MA)\\Modifica API endpoints\\Creazione auth. SDK (WS)\\Creazione default AC policies\\Creazione tests(WS)\\Creazione tests(MA)} & \makecell{Pietro Cipriani\\Elia e Pietro\\-\\Pietro Cipriani\\Elia Ziviani\\-\\-\\Elia Ziviani\\-} & \makecell{4\\6\\4\\2\\4\\3\\3\\4\\4} \\
        \hline
    \end{tabularx}
    \caption{Product backlog user story 3}
\end{table}

\clearpage

\subsubsection{Accesso alle criticità in città}
\textbf{User story}: Da utente voglio essere in grado di visualizzare la lista delle criticità presenti in città.\\
\begin{table}[htbp]
    \centering
    \renewcommand{\arraystretch}{1.3} % Imposta lo spazio verticale delle righe
    \begin{tabularx}{\textwidth}{| X | r | r | r | r |}
        \Xhline{2pt}
        \makecell{\textbf{Nome}} & \makecell{\textbf{User story}} & \makecell{\textbf{Cosa fare}} & \makecell{\textbf{Assegnazione}} & \makecell{\textbf{Stima}} \\
        \Xhline{2pt}
        \makecell{Accesso alle\\criticità in\\città} & \makecell{Da utente voglio essere\\in grado di visualizzare la\\lista delle criticità presenti\\in città} & \makecell{Istanziazione DB\\Creazione screen (MA)\\Creazione API endpoint (WS)\\Creazione metodo del DAO(WS)\\Creazione metodo del client(MA)\\Creazione event processor(WS)\\Creazione test(WS)\\Creazione test(MA)\\Istanziazione public server} & \makecell{Elia Ziviani\\Pietro Cipriani\\Dennis Orlando\\Dennis Orlando\\Pietro Cipriani\\Dennis Orlando\\Elia Ziviani\\-\\Elia Ziviani} & \makecell{2\\5\\4\\2\\2\\1\\5\\5\\-} \\
        \hline
    \end{tabularx}
    \caption{Product backlog user story 4}
\end{table}

\subsubsection{Visualizzazione criticità su mappa}
\textbf{User story}: Da utente voglio visualizzare su una cartina le zone colpite dai diversi eventi, in modo da capire visivamente se in una zona di mio interesse vi è un evento attivo.\\
\begin{table}[htbp]
    \centering
    \renewcommand{\arraystretch}{1.3} % Imposta lo spazio verticale delle righe
    \begin{tabularx}{\textwidth}{| X | r | r | r | r |}
        \Xhline{2pt}
        \makecell{\textbf{Nome}} & \makecell{\textbf{User story}} & \makecell{\textbf{Cosa fare}} & \makecell{\textbf{Assegnazione}} & \makecell{\textbf{Stima}} \\
        \Xhline{2pt}
        \makecell{Visualizzazione\\criticità\\su mappa} & \makecell{Da utente voglio\\visualizzare su una cartina\\le zone colpite dai diversi\\eventi, in modo da capire\\visivamente se in una zona\\di mio interesse vi è un\\evento attivo} & \makecell{Estensione screen (MA)\\Creazione tests (MA)} & \makecell{-\\-} & \makecell{3\\2} \\
        \hline
    \end{tabularx}
    \caption{Product backlog user story 5}
\end{table}

\clearpage

\subsection{Test cases}
\subsubsection{Test cases per il modulo \texttt{public\_server}}
Il modulo \texttt{public\_server} è il modulo che si occupa di gestire le richieste provenienti dall'applicativo mobile.\\
Infatti ha il compito di gestire tutte le richieste provenienti dagli applicativi mobile installati sui dispositivi, andando quindi a reperire dal database tutti gli eventi che sono richiesti nella richiesta stessa.\\

\begin{itemize}
    \item Test case per la richiesta di tutti gli eventi:
\end{itemize}

\begin{table}[htbp]
    \centering
    \renewcommand{\arraystretch}{1.3} % Imposta lo spazio verticale delle righe
    \begin{tabularx}{\textwidth}{| r | X | X | X | X | X | X |}
        \Xhline{2pt}
        \makecell{\textbf{No.}} & \makecell{\textbf{Descrizione}} & \makecell{\textbf{Dati}} & \makecell{\textbf{Precondizioni}} & \makecell{\textbf{Risultati attesi}} & \makecell{\textbf{Note}} \\
        \Xhline{2pt}
        1 & Test di successo per la richiesta di tutti gli eventi & Andare a \texttt{api/v3/events} & Il database dovrebbe contenere degli eventi per ottenerne una collezione & La richiesta viene eseguita con successo e vengono restituiti tutti gli eventi presenti - Codice: \texttt{200} & Nel caso in cui il database non contenga eventi, viene in ogni caso restituito il codice \texttt{200} con pero' una collezione da \texttt{0} elementi \\
        \hline
    \end{tabularx}
\end{table}

\clearpage

\begin{itemize}
    \item Test cases per la richiesta degli eventi utilizzando il parametro mode:
\end{itemize}

\begin{table}[htbp]
    \centering
    \renewcommand{\arraystretch}{1.3} % Imposta lo spazio verticale delle righe
    \begin{tabularx}{\textwidth}{| r | X | X | X | X | X | X |}
        \Xhline{2pt}
        \makecell{\textbf{No.}} & \makecell{\textbf{Descrizione}} & \makecell{\textbf{Dati}} & \makecell{\textbf{Precondizioni}} & \makecell{\textbf{Risultati attesi}} & \makecell{\textbf{Note}} \\
        \Xhline{2pt}
        2 & Test di successo per la richiesta di tutti gli eventi individuali con la sovrascrittura delle preferenze in locale con quelle in remoto & Andare a \texttt{/api/v3/events} ed impostare il parametro \texttt{mode=overwrite} & Il database dovrebbe contenere degli eventi che rispettino le preferenze locali del dispositivo che effettua la richiesta per ottenere una collezione popolata & La richiesta viene eseguita con successo e vengono restituiti tutti gli eventi presenti che rispecchiano le preferenze impostate in locale del dispositivo che effettua la richiesta - Codice: \texttt{200} & Se il database non contiene eventi o tutti non rispecchiano le preferenze locali, viene in ogni caso restituito il codice \texttt{200} con pero' una collezione da \texttt{0} elementi \\
        \hline
        3 & Test di successo per la richiesta di tutti gli eventi individuali combinando le preferenze in locale e in remoto & Andare a \texttt{/api/v3/events} ed impostare il parametro \texttt{mode=combine} & Il database dovrebbe contenere degli eventi che rispettino le preferenze sia locali che su server per ottenere una collezione popolata & La richiesta viene eseguita con successo e vengono restituiti tutti gli eventi presenti che rispecchiano le preferenze impostate in locale e su server - Codice: \texttt{200} & Se il database non contiene eventi o tutti non rispecchiano le preferenze locali e su server, viene in ogni caso restituito il codice \texttt{200} con pero' una collezione da \texttt{0} elementi \\
        \hline
        4 & Test di successo per la richiesta di tutti gli eventi individuali utilizzando i parametri in remoto in mancanza di quelli locali & Andare a \texttt{/api/v3/events} ed impostare il parametro \texttt{mode=ifempty} & Il database dovrebbe contenere degli eventi che rispettino le preferenze su server per ottenere una collezione popolata & La richiesta viene eseguita con successo e vengono restituiti tutti gli eventi presenti che rispecchiano le preferenze impostate su server - Codice: \texttt{200} & Se il database non contiene eventi o tutti non rispecchiano le preferenze su server, viene in ogni caso restituito il codice \texttt{200} con pero' una collezione da \texttt{0} elementi \\
        \hline
        5 & Test di fallimento per la richiesta di tutti gli eventi individuali con un parametro non valido & Andare a \texttt{/api/v3/events} ed impostare un parametro non valido & - & La richiesta non viene eseguita con successo e viene restituito un errore - Codice: \texttt{400} & - \\
        \hline
    \end{tabularx}
\end{table}

\clearpage

\begin{itemize}
    \item Test cases per la richiesta degli eventi utilizzando il parametro \textbf{addb}:
\end{itemize}
(addb: Lista di categorie (ID) da aggiungere ai filtraggi degli eventi broadcast)

\begin{table}[htbp]
    \centering
    \renewcommand{\arraystretch}{1.3} % Imposta lo spazio verticale delle righe
    \begin{tabularx}{\textwidth}{| r | X | X | X | X | X | X |}
        \Xhline{2pt}
        \makecell{\textbf{No.}} & \makecell{\textbf{Descrizione}} & \makecell{\textbf{Dati}} & \makecell{\textbf{Precondizioni}} & \makecell{\textbf{Risultati attesi}} & \makecell{\textbf{Note}} \\
        \Xhline{2pt}
        6 & Test di successo per la richiesta di tutti gli eventi appartenenti a categorie valide & Andare a \texttt{/api/v3/events} ed impostare il parametro \texttt{addb} con una lista di categorie valide (Numeri \texttt{> 0}) & Il database dovrebbe contenere degli eventi che appartengono alle categorie specificate per ottenere una collezione popolata & La richiesta viene eseguita con successo e vengono restituiti tutti gli eventi presenti che appartengono alle categorie specificate - Codice: \texttt{200} & Se il database non contiene eventi o tutti non appartengono alle categorie specificate, viene in ogni caso restituito il codice \texttt{200} con pero' una collezione da \texttt{0} elementi \\
        \hline
        7 & Test di fallimento per la richiesta di eventi appartenenti a categorie non valide (Non esistenti nel DB) & Andare a \texttt{/api/v3/events} ed impostare il parametro \texttt{addb} con una lista di categorie non valide (Non esistenti nel DB) & - & La richiesta non viene eseguita con successo e viene restituito un errore - Codice: \texttt{400} & - \\
        \hline
        8 & Test di fallimento per la richiesta di eventi appartenenti a categorie non accettabili (Numeri negativi) & Andare a \texttt{/api/v3/events} ed impostare il parametro \texttt{addb} con una lista di categorie minori di \texttt{0} & - & La richiesta non viene eseguita con successo e viene restituito un errore - Codice: \texttt{400} & - \\
        \hline
        9 & Test di fallimento per la richiesta di eventi appartenenti a categorie non accettabili (Non numeriche) & Andare a \texttt{/api/v3/events} ed impostare il parametro \texttt{addb} con una lista di categorie non valide (Non numeriche) & - & La richiesta non viene eseguita con successo e viene restituito un errore - Codice: \texttt{406} & - \\
        \hline
    \end{tabularx}
\end{table}

\clearpage

\begin{itemize}
    \item Test cases per la richiesta degli eventi utilizzando il parametro \textbf{subb}:
\end{itemize}
(subb: Lista di categorie (ID) da rimuovere dai filtraggi degli eventi broadcast)

\begin{table}[htbp]
    \centering
    \renewcommand{\arraystretch}{1.3}
    \begin{tabularx}{\textwidth}{| r | X | X | X | X | X | X |}
        \Xhline{2pt}
        \makecell{\textbf{No.}} & \makecell{\textbf{Descrizione}} & \makecell{\textbf{Dati}} & \makecell{\textbf{Precondizioni}} & \makecell{\textbf{Risultati attesi}} & \makecell{\textbf{Note}} \\
        \Xhline{2pt}
        10 & Test di successo per la richiesta di tutti gli eventi broadcast escludendo le categorie specificate & Andare a \texttt{/api/v3/events} ed impostare il parametro \texttt{subb} con una lista di categorie valide (Numeri \texttt{> 0}) & Il database dovrebbe contenere degli eventi che non appartengono alle categorie specificate per ottenere una collezione popolata & La richiesta viene eseguita con successo e vengono restituiti tutti gli eventi presenti che non appartengono alle categorie specificate - Codice: \texttt{200} & Se il database non contiene eventi o tutti appartengono alle categorie specificate, viene in ogni caso restituito il codice \texttt{200} con pero' una collezione da \texttt{0} elementi \\
        \hline
        11 & Test di fallimento per la richiesta di eventi broadcast escludendo categorie non valide (Non esistenti nel DB) & Andare a \texttt{/api/v3/events} ed impostare il parametro \texttt{subb} con una lista di categorie non valide (Non esistenti nel DB) & - & La richiesta non viene eseguita con successo e viene restituito un errore - Codice: \texttt{400} & Le categorie selezionate non esistono, sono ritornati tutti gli eventi broadcast nel DB \\
        \hline
        12 & Test di fallimento per la richiesta di eventi broadcast escludendo categorie non accettabili (Numeri negativi) & Andare a \texttt{/api/v3/events} ed impostare il parametro \texttt{subb} con una lista di categorie minori di \texttt{0} & - & La richiesta non viene eseguita con successo e viene restituito un errore - Codice: \texttt{400} & Le categorie selezionate non sono accettabili in quanto numeri negativi, sono ritornati tutti gli eventi broadcast nel DB \\
        \hline
        13 & Test di fallimento per la richiesta di eventi broadcast escludendo categorie non accettabili (Non numeriche) & Andare a \texttt{/api/v3/events} ed impostare il parametro \texttt{subb} con una lista di categorie non valide (Non numeriche) & - & La richiesta non viene eseguita con successo e viene restituito un errore - Codice: \texttt{406} & Le categorie selezionate non sono accettabili in quanto non numeriche, sono ritornati tutti gli eventi broadcast nel DB \\
        \hline
    \end{tabularx}
\end{table}

\clearpage

\begin{itemize}
    \item Test cases per la richiesta degli eventi utilizzando il parametro \textbf{addi}:
\end{itemize}
(addi: Lista di categorie (ID) da aggiungere ai filtraggi degli eventi individuali)

\begin{table}[htbp]
    \centering
    \renewcommand{\arraystretch}{1.3}
    \begin{tabularx}{\textwidth}{| r | X | X | X | X | X | X |}
        \Xhline{2pt}
        \makecell{\textbf{No.}} & \makecell{\textbf{Descrizione}} & \makecell{\textbf{Dati}} & \makecell{\textbf{Precondizioni}} & \makecell{\textbf{Risultati attesi}} & \makecell{\textbf{Note}} \\
        \Xhline{2pt}
        14 & Test di successo per la richiesta di tutti gli eventi individuali appartenenti a categorie valide & Andare a \texttt{/api/v3/events} ed impostare il parametro \texttt{addi} con una lista di categorie valide (Numeri \texttt{> 0}) & Il database dovrebbe contenere degli eventi individuali che appartengono alle categorie specificate per ottenere una collezione popolata & La richiesta viene eseguita con successo e vengono restituiti tutti gli eventi presenti che appartengono alle categorie specificate - Codice: \texttt{200} & Se il database non contiene eventi o tutti non appartengono alle categorie specificate, viene in ogni caso restituito il codice \texttt{200} con pero' una collezione da \texttt{0} elementi \\
        \hline
        15 & Test di fallimento per la richiesta di eventi individuali appartenenti a categorie non valide (Non esistenti nel DB) & Andare a \texttt{/api/v3/events} ed impostare il parametro \texttt{addi} con una lista di categorie non valide (Non esistenti nel DB) & - & La richiesta non viene eseguita con successo e viene restituito un errore - Codice: \texttt{400} & - \\
        \hline
        16 & Test di fallimento per la richiesta di eventi individuali appartenenti a categorie non accettabili (Numeri negativi) & Andare a \texttt{/api/v3/events} ed impostare il parametro \texttt{addi} con una lista di categorie minori di \texttt{0} & - & La richiesta non viene eseguita con successo e viene restituito un errore - Codice: \texttt{400} & - \\
        \hline
        17 & Test di fallimento per la richiesta di eventi individuali appartenenti a categorie non accettabili (Non numeriche) & Andare a \texttt{/api/v3/events} ed impostare il parametro \texttt{addi} con una lista di categorie non valide (Non numeriche) & - & La richiesta non viene eseguita con successo e viene restituito un errore - Codice: \texttt{406} & - \\
        \hline
    \end{tabularx}
\end{table}

\clearpage

\begin{itemize}
    \item Test cases per la richiesta degli eventi utilizzando il parametro \textbf{subi}:
\end{itemize}
(subi: Lista di categorie (ID) da rimuovere dai filtraggi degli eventi individuali)

\begin{table}[htbp]
    \centering
    \renewcommand{\arraystretch}{1.3}
    \begin{tabularx}{\textwidth}{| r | X | X | X | X | X | X |}
        \Xhline{2pt}
        \makecell{\textbf{No.}} & \makecell{\textbf{Descrizione}} & \makecell{\textbf{Dati}} & \makecell{\textbf{Precondizioni}} & \makecell{\textbf{Risultati attesi}} & \makecell{\textbf{Note}} \\
        \Xhline{2pt}
        18 & Test di successo per la richiesta di tutti gli eventi individuali escludendo le categorie specificate & Andare a \texttt{/api/v3/events} ed impostare il parametro \texttt{subi} con una lista di categorie valide (Numeri \texttt{> 0}) & Il database dovrebbe contenere degli eventi individuali che non appartengono alle categorie specificate per ottenere una collezione popolata & La richiesta viene eseguita con successo e vengono restituiti tutti gli eventi presenti che non appartengono alle categorie specificate - Codice: \texttt{200} & Se il database non contiene eventi o tutti appartengono alle categorie specificate, viene in ogni caso restituito il codice \texttt{200} con pero' una collezione da \texttt{0} elementi \\
        \hline
        19 & Test di fallimento per la richiesta di eventi individuali escludendo categorie non valide (Non esistenti nel DB) & Andare a \texttt{/api/v3/events} ed impostare il parametro \texttt{subi} con una lista di categorie non valide (Non esistenti nel DB) & - & La richiesta non viene eseguita con successo e viene restituito un errore - Codice: \texttt{400} & Le categorie selezionate non esistono, sono ritornati tutti gli eventi broadcast nel DB \\
        \hline
        20 & Test di fallimento per la richiesta di eventi individuali escludendo categorie non accettabili (Numeri negativi) & Andare a \texttt{/api/v3/events} ed impostare il parametro \texttt{subi} con una lista di categorie minori di \texttt{0} & - & La richiesta non viene eseguita con successo e viene restituito un errore - Codice: \texttt{400} & Le categorie selezionate non sono accettabili in quanto numeri negativi, sono ritornati tutti gli eventi broadcast nel DB \\
        \hline
        21 & Test di fallimento per la richiesta di eventi individuali escludendo categorie non accettabili (Non numeriche) & Andare a \texttt{/api/v3/events} ed impostare il parametro \texttt{subi} con una lista di categorie non valide (Non numeriche) & - & La richiesta non viene eseguita con successo e viene restituito un errore - Codice: \texttt{406} & Le categorie selezionate non sono accettabili in quanto non numeriche, sono ritornati tutti gli eventi broadcast nel DB \\
        \hline
    \end{tabularx}
\end{table}

\clearpage

\begin{itemize}
    \item Test cases per la richiesta di un evento specifico utilizzando il parametro \textbf{id}:
\end{itemize}

\begin{table}[htbp]
    \centering
    \renewcommand{\arraystretch}{1.3}
    \begin{tabularx}{\textwidth}{| r | X | X | X | X | X | X |}
        \Xhline{2pt}
        \makecell{\textbf{No.}} & \makecell{\textbf{Descrizione}} & \makecell{\textbf{Dati}} & \makecell{\textbf{Precondizioni}} & \makecell{\textbf{Risultati attesi}} & \makecell{\textbf{Note}} \\
        \Xhline{2pt}
        22 & Test di successo per la richiesta di un evento specifico & Andare a \texttt{/api/v3/events} ed impostare il parametro \texttt{id} con un ID valido & Il database dovrebbe contenere un evento con l'ID specificato per ottenere un evento specifico & La richiesta viene eseguita con successo e viene restituito l'evento specifico richiesto - Codice: \texttt{200} & Se il database non contiene un evento con l'ID specificato, viene restituito il codice \texttt{404} \\
        \hline
        23 & Test di fallimento per la richiesta di un evento specifico con un ID non valido (Numero negativo) & Andare a \texttt{/api/v3/events} ed impostare il parametro \texttt{id} con un ID non valido (Numero negativo) & - & La richiesta non viene eseguita con successo e viene restituito un errore - Codice: \texttt{406} & - \\
        \hline
        24 & Test di fallimento per la richiesta di un evento specifico con un ID non valido (Non numerico) & Andare a \texttt{/api/v3/events} ed impostare il parametro \texttt{id} con un ID non valido (Non numerico) & - & La richiesta non viene eseguita con successo e viene restituito un errore - Codice: \texttt{406} & - \\
        \hline
        25 & Test di fallimento per la richiesta di un evento specifico con un ID non esistente nel DB & Andare a \texttt{/api/v3/events} ed impostare il parametro \texttt{id} con un ID non esistente nel DB & - & La richiesta non viene eseguita con successo e viene restituito un errore - Codice: \texttt{404} & - \\
        \hline
    \end{tabularx}
\end{table}

\clearpage

\subsubsection{Test cases per il modulo \texttt{management\_server}}

Il modulo, per questo primo sprint, si occupa dell'inserimento di un nuovo evento all'interno del sistema da parte degli utenti autorizzati, in modo che possa essere visibile dall'applicazione mobile degli utenti autenticati e non.\\

\begin{table}[htbp]
    \centering
    \renewcommand{\arraystretch}{1.3}
    \begin{tabularx}{\textwidth}{| r | X | X | X | X | X | X |}
        \Xhline{2pt}
        \makecell{\textbf{No.}} & \makecell{\textbf{Descrizione}} & \makecell{\textbf{Dati}} & \makecell{\textbf{Precondizioni}} & \makecell{\textbf{Risultati attesi}} & \makecell{\textbf{Note}} \\
        \Xhline{2pt}
        1 & Test di successo per l'inserzione di un nuovo evento & Andare a \texttt{/api/v3/} \texttt{insert\_new\_event} e fornire in formato bson i dati dell'evento da inserire & - & L'evento viene inserito con successo nel database - Codice: \texttt{200} & - \\
        \hline
        2 & Test di fallimento per l'inserzione di un nuovo evento con dati non validi & Andare a \texttt{/api/v3/} \texttt{insert\_new\_event} e fornire in formato bson i dati dell'evento da inserire con dati non validi & - & L'evento non viene inserito con successo nel database e viene restituito un errore - Codice: \texttt{400} & Devono essere rispettati tutti i tipi per ogni caratteristica dell'evento \\
        \hline
        3 & Test di fallimento per l'inserzione di un nuovo evento con parametro \texttt{id} uguale ad un evento già esistente & Andare a \texttt{/api/v3/} \texttt{insert\_new\_event} e fornire in formato bson i dati dell'evento da inserire con il parametro \texttt{id} uguale ad un evento già esistente & - & L'evento non viene inserito con successo nel database e viene restituito un errore - Codice: \texttt{406} & - \\
        \hline
    \end{tabularx}
\end{table}

\clearpage

\subsection{Sprint review}
Commenti su aspetti del meeting: demo e discussione successiva

\subsection{Product backlog refinement}
Commenti su aspetti del meeting e variazioni al product backlog

\subsection{Sprint retrospective}
Commenti sulle conclusioni del meeting, cosa ha funzionato, cosa no, adozione di pratiche "agile", "design thinking", dinamiche di gruppo, spunti per il prossimo sprint, etc.

\end{document}
